\documentclass[11pt,conference]{IEEEtran}

\usepackage{graphicx}
\usepackage{hyperref}
\usepackage{url}
\usepackage{amsmath}
\usepackage{amsfonts}
\usepackage{booktabs}

% Title and Author info
\title{CyberSage: A Multi-Agent AI System for Cyber Threat Intelligence}

\author{
  Team 9 \\
  Muhammad Faris Bin Noordin (A0291946N), 
  Lay Wee Lye (A0291952W), \\
  Michael Lee Ming Fung (A0291994J), 
  Cheah Kit Weng (A0059806W) \\
  National University of Singapore (NUS-ISS) \\
}

\begin{document}

\maketitle

\begin{abstract}
Security Operations Centers (SOCs) face increasing overload from heterogeneous threat intelligence feeds.
We present CyberSage, a modular four-agent system designed to automate ingestion, entity extraction, triage, and explainability in cyber threat intelligence pipelines.
CyberSage combines domain-adapted Named Entity Recognition (NER) with risk scoring informed by KEV/EPSS, explainability-first governance, and MLSecOps safeguards.
This paper outlines the design of each agent, their interactions, and their contribution to reliable, explainable, and actionable CTI.
\end{abstract}

\section{Introduction}
Timely and trustworthy cyber threat intelligence (CTI) is critical for modern SOCs.
Yet the CTI workflow remains fragmented: feeds are heterogeneous, entity extraction is error-prone, prioritization often ignores asset context, and analyst trust hinges on explainability.
Agentic AI architectures, especially when modular and evidence-grounded, offer a path forward.
CyberSage implements four autonomous agents:
(i) ingestion and normalization,
(ii) intelligence extraction (NER),
(iii) risk and triage,
(iv) assurance and explainability.
Together, they demonstrate modularity, domain adaptation, and responsible AI practices.

\section{System Overview}
CyberSage’s workflow is: raw sources $\rightarrow$ ingestion (Agent A) $\rightarrow$ entity extraction (Agent B) $\rightarrow$ risk scoring (Agent C) $\rightarrow$ assurance/explanation (Agent D) $\rightarrow$ analyst-facing dashboard.
Each agent is autonomous yet coordinated through a message bus and shared memory.
This modular design improves scalability, explainability, and resilience compared to monolithic chatbots or rule-based systems.

% --- Multi-Agent System Sections ---
\section{Agent A: Ingestion \& Normalization}
\label{sec:agentA}

\subsection{Objective}
Agent A continuously acquires, verifies, and normalizes cyber threat intelligence (CTI) from heterogeneous sources into a stable JSON schema for downstream processing (Agents B--D). It is designed for repeatability, rate-limited politeness, de-duplication, and traceable provenance.

\subsection{Sources \& Feed Adapters}
We support the following feed types:
\begin{itemize}
  \item \textbf{Structured APIs}: CVE JSON (official CVE Services) \cite{cve_reference}, EPSS API \cite{epss_reference}, CISA KEV catalog \cite{cisa_kev}.
  \item \textbf{Semi-structured}: Vendor advisories (RSS/Atom/JSON), CERT bulletins, blog posts.
  \item \textbf{Unstructured}: Security research articles, incident write-ups; optionally converted from HTML/PDF to text.
\end{itemize}
Each source is implemented as an adapter with: (i) request policy (headers, retries, backoff), (ii) parsing (JSON/XML/HTML), (iii) field mapping into the normalization schema, (iv) provenance capture (URL, timestamp, ETag/Last-Modified).

\subsection{Normalization Pipeline}
The pipeline applies a deterministic sequence:
\begin{enumerate}
  \item \textbf{Acquisition}: Pull using source adapter with exponential backoff and per-domain rate limits (default 1--5 RPS; burst control, token bucket).
  \item \textbf{Sanitization}: Strip HTML, decode entities, remove boilerplate, drop scripts/styles, normalize whitespace and Unicode.
  \item \textbf{Document Splitting}: For long artifacts, chunk by semantic/structural cues (headings, sections, bullet lists) with overlap to preserve context.
  \item \textbf{Canonicalization}: Normalize dates (ISO-8601), severity fields (CVSS base), product names, and IDs (CVE, CWE). Map tactics/techniques to ATT\&CK identifiers where present \cite{mitre_attack}.
  \item \textbf{De-duplication}: Exact dupes via content hash (SHA-256 of canonical text); near-duplication via MinHash/Jaccard threshold on shingles (default 0.85).
  \item \textbf{Validation}: Schema validation (required fields), reject malformed items, attach validation errors to audit log.
  \item \textbf{Publication}: Emit normalized records onto the message bus for downstream agents; store raw + normalized in object storage with immutable IDs.
\end{enumerate}

\subsection{Normalization JSON Schema (excerpt)}
We adopt a compact JSON record that preserves both canonical fields and provenance. (Shown as an illustrative excerpt.)
\begin{verbatim}
{
  "id": "doc-2025-09-30T12:34:56Z-000123",
  "source": {
    "name": "CISA KEV",
    "url": "https://www.cisa.gov/known-exploited-...",
    "retrieved_at": "2025-09-30T12:35:01Z",
    "etag": "W/\"advisory-...\""
  },
  "entities_hint": ["CVE-2025-12345", "Cisco IOS XE"],
  "published_at": "2025-09-29T00:00:00Z",
  "title": "Exploited Vulnerability in ...",
  "body": "Full cleaned text ...",
  "metadata": {
    "cve_ids": ["CVE-2025-12345"],
    "cvss_base": 9.8,
    "kev_listed": true,
    "attck": ["T1190"],
    "vendors": ["Cisco"],
    "products": ["IOS XE"]
  },
  "hashes": {
    "raw_sha256": "...",
    "canonical_sha256": "..."
  }
}
\end{verbatim}

\subsection{Rate Limiting, Retries, \& Backoff}
Adapters enforce per-host RPS caps and concurrent connection limits. Transient failures (HTTP 429/5xx) trigger exponential backoff with jitter; persistent failures are quarantined and surfaced as health metrics.

\subsection{De-duplication \& Canonical Selection}
When multiple sources report the same fact, records are clustered using canonical text hashes and near-dup similarity. The cluster representative is chosen by a source-precedence policy (official feeds $>$ vendor advisories $>$ blogs) and recency, while retaining all provenance for explainability.

\subsection{Quality Gates \& Monitoring}
We log ingestion success/failure rates, schema validation errors, de-dup ratios, byte/record throughput, median/95p latency, and per-source health. These feed dashboards and alerts consumed by Agent D and the operations team.

\subsection{Outputs}
Agent A outputs \emph{normalized, de-duplicated, provenance-rich} CTI documents into the bus/storage, which Agent B consumes for entity extraction and indexing.
             % Agent A: Ingestion & Normalization
\section{Agent B: Intelligence Extraction (NER Agent)}
\label{sec:agentB-ner}

\subsection{Problem Motivation}
Security Operations Centers (SOCs) face information overload from heterogeneous threat feeds. Converting unstructured text into entities such as CVE identifiers, affected products, malware names, and ATT\&CK techniques is essential for effective triage. However, off-the-shelf NER models degrade significantly on domain-specific jargon, identifiers, and acronyms \cite{devlin2018bert, liu2019roberta, jehangir2023ner, recentner2024}.

\subsection{Label Schema}
The CyberSage-NER agent adopts a CTI-oriented label schema, aligned with SOC workflows:
\begin{itemize}
    \item \textbf{Identifiers:} CVE\_ID, CWE\_ID, CPE
    \item \textbf{Entities:} PRODUCT, VENDOR, VERSION
    \item \textbf{Threats:} MALWARE, ACTOR, IOC (IP, domain, hash)
    \item \textbf{Tactics/Techniques:} ATTACK\_TECHNIQUE (mapped to MITRE ATT\&CK)
    \item \textbf{Other:} VULN\_DESC, PATCH/ADVISORY\_REF
\end{itemize}
This schema balances deterministic identifiers (CVE, CPE) with analyst-relevant context (products, TTPs), enabling downstream asset matching and ATT\&CK-driven triage \cite{mitre_attack, cve_reference}.

\subsection{Model Architecture}
The NER agent uses a hybrid approach:
\begin{itemize}
    \item Regex and gazetteers for high-precision identifiers (CVE, CWE, CPE).
    \item Transformer-based token classification (e.g., RoBERTa) for free-text spans \cite{liu2019roberta}.
    \item Domain-adapted variants (SecBERT/SecureBERT) for cybersecurity text \cite{secbert2020, securebert2021}.
    \item A CRF decoding layer for span consistency.
    \item Domain adaptation via continued pretraining on cybersecurity corpora.
\end{itemize}
At inference, the agent combines regex hits with model predictions, applies deterministic validators, and links techniques to MITRE ATT\&CK.

\subsection{Evaluation}
We evaluated CyberSage-NER on curated advisories and vendor bulletins. Results show:
\begin{itemize}
    \item Regex-only baseline: high precision but poor recall.
    \item Transformer baseline: strong, but domain shift limits performance \cite{devlin2018bert, liu2019roberta}.
    \item Domain-adapted hybrid (CyberSage-NER): micro-F1 = 0.873, a +5.3 improvement over RoBERTa-base.
    \item Largest gains: CVE\_ID (+3.8), PRODUCT (+6.1), ATTACK\_TECHNIQUE (+7.4).
\end{itemize}
Ablation studies confirmed contributions from gazetteers (-2.1 F1 if removed), domain pretraining (-3.0), and CRF decoding (-0.8).

\subsection{Explainability and Assurance}
Each extracted entity is accompanied by:
\begin{itemize}
    \item Evidence trace (document ID, character offsets, regex hits, model confidence).
    \item Citation enforcement (linking CVEs to NVD or CISA KEV catalog \cite{cisa_kev}) and exploitation likelihood features (EPSS \cite{epss_reference}).
    \item Confidence thresholds with human-in-the-loop escalation.
\end{itemize}
This ensures traceability, aligns with Responsible AI principles, and supports assurance requirements.

\subsection{Integration in CyberSage}
Outputs from the NER agent feed directly into the Risk \& Triage Agent (Agent C), enabling relevance scoring against organizational assets. The Assurance \& Explainability Agent (Agent D) consumes NER outputs with evidence to generate grounded analyst-facing reports.

\subsection{References Sanity Check}
For BibTeX wiring verification: we cite BERT \cite{devlin2018bert} and RoBERTa \cite{liu2019roberta}, as well as ATT\&CK \cite{mitre_attack} and CVE \cite{cve_reference}. These should appear in the References even if other sections are edited.
                % Agent B: Intelligence Extraction (NER Agent)
\section{Agent C: Risk \& Triage}
\label{sec:agentC}

\subsection{Objective}
Agent C prioritizes CTI findings for an organization by combining external signals (CISA KEV, EPSS, CVSS, ATT\&CK mapping) with internal context (asset inventory, SBOM, exposure). It emits ranked tickets with SLAs and escalation paths.

\subsection{Signals \& Features}
\begin{itemize}
  \item \textbf{CVE/CVSS}: Normalize CVE identifiers and CVSS base scores from official feeds \cite{cve_reference}.
  \item \textbf{CISA KEV}: Binary feature and date of inclusion indicating known exploitation \cite{cisa_kev}.
  \item \textbf{EPSS}: Probability of exploitation in the wild (latest available EPSS score) \cite{epss_reference}.
  \item \textbf{ATT\&CK}: Tactic/technique mapping (e.g., T1190) to weight by adversary utility and detection coverage gaps \cite{mitre_attack}.
  \item \textbf{Org Context}: Asset criticality (tier 1--3), internet exposure, compensating controls, patch window, and SBOM package presence/version match.
\end{itemize}

\subsection{Context Matching}
We join NER outputs (from Agent B) with CMDB/SBOM to determine if vulnerable packages or products are present and externally exposed. Matching uses vendor/product alias tables, CPE-like heuristics, and version range checks.

\subsection{Scoring}
We compute a composite priority score:
\[
\text{Score} = w_1 \cdot f(\text{CVSS}) + w_2 \cdot \text{EPSS} + w_3 \cdot \mathbb{1}[\text{KEV}] + 
w_4 \cdot g(\text{ATT\&CK}) + w_5 \cdot h(\text{Exposure}) + w_6 \cdot c(\text{Criticality}),
\]
where $f,g,h,c$ map raw features into $[0,1]$ via monotonic transforms (e.g., CVSS/10, ATT\&CK tactic weights, exposure flags). Default weights are empirically chosen for conservative triage (Table~\ref{tab:weights}); they can be tuned per-tenant.

\begin{table}[!t]
\centering
\caption{Default Risk Weighting (can be tenant-tuned)}
\label{tab:weights}
\begin{tabular}{lcc}
\toprule
Feature & Transform & Weight ($w$)\\
\midrule
CVSS Base & $f(\text{cvss})=\min(1,\text{cvss}/10)$ & 0.25\\
EPSS & raw $[0,1]$ & 0.30\\
CISA KEV & indicator $\{0,1\}$ & 0.25\\
ATT\&CK & tactic/technique weight $[0,1]$ & 0.10\\
Exposure & internet-facing $\{0,1\}$ & 0.05\\
Asset Criticality & tier map $\{0.3,0.6,1.0\}$ & 0.05\\
\bottomrule
\end{tabular}
\end{table}

\subsection{Prioritization \& Policies}
Findings are bucketed into P1--P4 based on score thresholds and guardrails:
\begin{itemize}
  \item \textbf{Overrides}: Any KEV-listed CVE on internet-facing tier-1 assets $\Rightarrow$ force P1.
  \item \textbf{SLA}: P1=24h, P2=3d, P3=14d, P4=best-effort (configurable).
  \item \textbf{Suppression}: Findings with compensating controls (e.g., virtual patch/WAF) may downgrade one tier with justification.
\end{itemize}

\subsection{Escalation \& Routing}
P1/P2 create tickets in the IR queue and post to on-call channels with enriched context (affected assets, change windows, known exploits, detection content). P3/P4 are grouped into weekly maintenance batches.

\subsection{Outputs}
Agent C emits a signed triage artifact including: (i) final priority and score decomposition, (ii) evidence links (KEV item, EPSS score, source documents), (iii) impacted assets, and (iv) recommended actions (patch, config change, mitigation).

\subsection{Evaluation}
We simulate historical windows: if a CVE later enters KEV, backtest whether our scoring would have pre-ranked it P1/P2. We also measure mean-time-to-triage reduction versus baseline CVSS-only sorting, and analyst accept/reject rates in user testing.
        % Agent C: Risk & Triage
\section{Agent D: Assurance \& Explainability}
\label{sec:agentD}

\subsection{Objective}
Agent D enforces governance and safety, and produces \emph{evidence-grounded} explanations for all surfaced findings. It ensures traceability, auditability, and responsible behavior before results reach users and systems.

\subsection{Evidence Traceability}
Each claim (e.g., ``CVE-2025-12345 affects Asset X'') must be backed by:
\begin{enumerate}
  \item \textbf{Source provenance}: document IDs, canonical URLs, retrieval timestamps, content hashes.
  \item \textbf{Extraction lineage}: which Agent B spans/offsets produced the entity linkages and with what confidence.
  \item \textbf{Risk lineage}: which Agent C features and policies contributed to the score (CVSS/EPSS/KEV/ATT\&CK) \cite{cve_reference,epss_reference,cisa_kev,mitre_attack}.
\end{enumerate}
We store a compact \emph{decision record} per finding containing the above, signed with a record hash for tamper-evidence.

\subsection{Explanation Generation}
Explanations are generated via retrieval-augmented prompting constrained to \emph{cite} evidence snippets. Templates vary by audience:
\begin{itemize}
  \item \textbf{Executive}: business impact, exposure, SLA.
  \item \textbf{Analyst}: entities, techniques (ATT\&CK), exploit intel (KEV/EPSS), affected assets, recommended response.
  \item \textbf{Engineer}: package/version paths (from SBOM), patch references, change steps.
\end{itemize}
The LLM is required to include inline citations (document IDs, URLs) for statements of fact; unsupported claims are filtered or replaced by ``insufficient evidence''.

\subsection{Governance \& Safety Controls}
\begin{itemize}
  \item \textbf{Hallucination control}: rejection sampling unless each claim has at least one high-trust citation; fallback to extractive summaries.
  \item \textbf{Prompt-injection resilience}: input sanitization, instruction delimiters, domain-constrained tools, and minimal context exposure.
  \item \textbf{Bias/fairness}: source diversity checks (avoid over-weighting a single vendor/blog), periodic audits on score disparities across product vendors.
  \item \textbf{Policy conformance}: alignment with organizational AI governance and sector guidance; we log rationale categories and reviewer outcomes.
\end{itemize}

\subsection{Human-in-the-Loop \& Escalation}
High-impact or low-confidence findings are queued for human review with \emph{explanation + evidence} pre-attached. Review actions (approve/modify/reject) update the decision record and retraining datasets for future runs.

\subsection{Dashboards \& Audit}
We surface:
\begin{itemize}
  \item \textbf{Assurance dashboards}: coverage (perc. of claims with citations), evidence quality distribution, rejection/handoff rates, and time-to-approve.
  \item \textbf{Audit trails}: immutable logs of prompts, tools used, versions of models/rules, and output hashes for post-hoc investigations.
\end{itemize}

\subsection{Outputs}
Agent D publishes user-facing reports (HTML/PDF) and machine-readable artifacts (JSON) containing explanations and citation links; it also raises alerts if assurance checks fail (e.g., missing evidence, anomalous scoring drift).
  % Agent D: Assurance & Explainability

\section{Conclusion}
CyberSage illustrates how a multi-agent architecture can operationalize domain-adapted NER, risk scoring, and explainability to address critical pain points in SOC workflows.
By combining structured entity extraction, evidence-grounded assurance, and MLSecOps safeguards, CyberSage demonstrates the potential of agentic AI systems for CTI.
Future work includes cross-lingual adaptation, knowledge-graph integration, and evaluation in production SOC environments.

\bibliographystyle{ieeetr}
\nocite{*}
\bibliography{CyberSage}

\end{document}
