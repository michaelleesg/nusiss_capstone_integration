documentclass[11pt,conference]{IEEEtran}

% Encoding & fonts
usepackage[utf8]{inputenc}
usepackage[T1]{fontenc}
usepackage[varqu]{zi4} % Inconsolata (tt), safer than manual shape hacks

% Core packages
usepackage{graphicx}
usepackage{amsmath}
usepackage{amsfonts}
usepackage{booktabs}
usepackage{listings}
usepackage{xcolor}
usepackage{cite} % IEEE numeric citations sortedcompressed

% Captions (IEEE-friendly)
usepackage[font=footnotesize,labelfont=bf]{caption}
captionsetup{skip=6pt}
captionsetup[lstlisting]{skip=0.8em}

% Optional compact headings (scope only to subsection)
usepackage[compact,explicit]{titlesec}
titlespacing{subsection}{0pt}{1.0ex plus 0.2ex minus .2ex}{0.6ex plus .1ex}
titleformat{subsection}[runin]{bfseries}{thesubsection}{0.5em}{}[]

% Hyperref MUST be last
usepackage[hidelinks]{hyperref}

% -------- Listings defaults --------
lstset{
	basicstyle=ttfamilyfootnotesize,
	breaklines=true,
	frame=single,
	columns=fullflexible,
	keepspaces=true,
	showstringspaces=false
}
% JSON language (minimal, ASCII-safe)
lstdefinelanguage{json}{
	basicstyle=ttfamilyfootnotesize,
	showstringspaces=false,
	breaklines=true,
	frame=single,
	morestring=[b],
	morecomment=[l]{},
	morecomment=[s]{}{},
}

% ---------- Title ----------
title{Individual Progress Report CyberSage}
author{
	Michael Lee Ming Fung (A0291994J)
	National University of Singapore (NUS-ISS)
}
date{}

begin{document}
maketitle

begin{abstract}
This report summarizes my individual contributions to the CyberSage project, a multi-agent AI system for cyber threat intelligence. I describe the scope of work undertaken, key design decisions, implementation artifacts, and evaluation results relevant to my assigned components. I also reflect on limitations and propose next steps. A one-page reproducibility appendix is included to enable an end-to-end slice (HEVA storage, retrieval, and Agent~C triage artifact emission) that mirrors the project’s management and technical report outputs.
end{abstract}

section{Introduction}
Briefly restate the overall project problem and the CyberSage approach (A~textrightarrow~B~textrightarrow~C~textrightarrow~D). Clarify your role and the components you owned or co-owned.

section{Objectives}
List the specific objectives you targeted (e.g., data contracts, retrieval quality, triage scoring, reporting).

section{Design and Implementation}
Describe the design choices you made and why. Include schemas, interfaces, and any toolslibraries used (e.g., Qdrant, FastAPI, spaCy, SentenceTransformer).

section{Experiments and Results}
Summarize your experiments, datasets, and results (tablesfigures if applicable). State limitations and error modes candidly.

section{Discussion}
Interpret results, lessons learned, and trade-offs. Note how your work integrated with other agents and where future improvements should focus.

section{Conclusion}
Concise wrap-up of contributions and impact on the overall system.

% -------- Optional references (reuse team .bib if desired) --------
% bibliographystyle{ieeetr}
% bibliography{CyberSage}

% ===================== APPENDIX =====================
section{Appendix Reproducibility (One Page)}
label{apprepro}

textbf{Goal.} Reproduce an end-to-end slice (i) stand up HEVA store; (ii) load minimal evidence for the Veeam incident; (iii) emit an Agent~C triage artifact that matches the Agent~C emph{management} and emph{technical} reportsfootnote{See texttt{management_report_20251004_030351.txt} and texttt{technical_report_20251004_030351.txt}.} (Medium 5.010, 4-round debate, moderate consensus).

subsection{Environment}
begin{lstlisting}[language=bash]
python -m venv .venv && source .venvbinactivate
pip install qdrant-client sentence-transformers fastapi uvicorn
# If behind a firewall, pre-download sentence-transformersall-MiniLM-L6-v2
end{lstlisting}

subsection{Start Vector DB (Qdrant)}
begin{lstlisting}[language=bash]
docker run -d --name qdrant --restart unless-stopped 
-p 63336333 -v qdrant-volqdrantstorage qdrantqdrantlatest
end{lstlisting}

subsection{Create Collection (384-dim, cosine)}
begin{lstlisting}[language=python]
# create_collection.py
from qdrant_client import QdrantClient
from qdrant_client.http.models import Distance, VectorParams
QdrantClient(host=localhost, port=6333).recreate_collection(
collection_name=ner_vectors,
vectors_config=VectorParams(size=384, distance=Distance.COSINE),
)
print(OK)
end{lstlisting}

subsection{Populate HEVA Evidence (from incident JSON)}
begin{lstlisting}[language=python]
# upsert_veeam_evidence.py
from qdrant_client import QdrantClient
from qdrant_client.http.models import PointStruct
from sentence_transformers import SentenceTransformer
import json, uuid, pathlib

p = pathlib.Path(vulnerability_analysis_results_20251004_030351.json)
J = json.loads(p.read_text(encoding=utf-8))
sel = J[selected_incident][assessment]  # title, summary, cve_vulns

sentences = [
f'{sel[title]} {sel[summary]}',
CVE-2025-23121 allows authenticated RCE on domain-joined VBR servers.,
Patch available in VBR 12.3.2.3617; active exploitation reported.
]

model = SentenceTransformer(sentence-transformersall-MiniLM-L6-v2)
vecs = model.encode(sentences, normalize_embeddings=True)

qc = QdrantClient(host=localhost, port=6333)
points = [PointStruct(id=str(uuid.uuid4()), vector=v.tolist(),
payload={text t, tags[DEMO,CVE]}) for t, v in zip(sentences, vecs)]
qc.upsert(collection_name=ner_vectors, points=points)
print(fOK upserted {len(points)})
end{lstlisting}

subsection{Verify Retrieval (direct search)}
begin{lstlisting}[language=python]
# demo_search_qdrant.py
from qdrant_client import QdrantClient
from sentence_transformers import SentenceTransformer

q = Evidence of exploitation for CVE-2025-23121
model = SentenceTransformer(sentence-transformersall-MiniLM-L6-v2)
qvec = model.encode([q], normalize_embeddings=True)[0]

qc = QdrantClient(host=localhost, port=6333)
hits = qc.search(collection_name=ner_vectors, query_vector=qvec, limit=5)
for h in hits
print(round(h.score,3), h.payload.get(text))
end{lstlisting}

subsection{Emit a Triage Artifact (values match Agent~C reports)}
begin{lstlisting}[language=python]
# emit_triage_artifact.py
import uuid, json, time
artifact = {
	id str(uuid.uuid4()),
	created_at time.strftime(%Y-%m-%dT%H%M%SZ, time.gmtime()),
	source {
		doc_id doc-2025-10-04-veeam,
		url httpswww.bleepingcomputer.comnewssecuritynew-veeam-rce-flaw-lets-domain-users-hack-backup-servers,
		title New Veeam RCE flaw lets domain users hack backup servers,
		source_type html,
		cve_vulns [CVE-2024-40711,CVE-2025-23120,CVE-2025-23121]
	},
	scores {
		overall_10 5.0, overall 0.50,
		components {
			exploitation_probability {score 0.60, weight 0.25},
			technical_severity      {score 0.40, weight 0.25},
			business_analysis       {score 0.50, weight 0.25},
			novelty_detection       {score 0.50, weight 0.25}
		}
	},
	priority P2,
	consensus {achieved True, strength Moderate,
		rounds_conducted 4, score_range 0.20, threshold 0.25},
	evidence [{
		text CVE-2025-23121 RCE on VBR; active exploitation; patch 12.3.2.3617.,
		score 0.90
	}]
}
print(json.dumps(artifact, indent=2))
end{lstlisting}

subsection{Run sequence}
begin{lstlisting}[language=bash]
python create_collection.py
python upsert_veeam_evidence.py
python demo_search_qdrant.py
python emit_triage_artifact.py  triage_artifact.json
end{lstlisting}

noindenttextbf{Expected.} Search returns the inserted sentences with high cosine scores. The emitted artifact shows Medium (5.010) with four equal-weight components and Moderate consensus over 4 rounds—matching the Agent~C management and technical reports. The artifact is the contract consumed by Agent~D.

end{document}
